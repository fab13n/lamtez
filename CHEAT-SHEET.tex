\documentclass[10pt,landscape,a4paper]{article}
\usepackage[utf8]{inputenc}
%\usepackage[nosf]{kpfonts}
%\usepackage[t1]{sourcesanspro}
\usepackage{multicol}
%\usepackage{wrapfig}
\usepackage[top=5mm,bottom=5mm,left=5mm,right=5mm]{geometry}
%\usepackage{microtype}
\usepackage{fancyvrb}

\usepackage{listings}% http://ctan.org/pkg/listings
\lstset{basicstyle=\ttfamily, mathescape}

\let\bar\overline
\renewcommand{\baselinestretch}{.8}
\pagestyle{empty}
\makeatletter

\setlength{\parindent}{0pt}

\renewcommand{\section}{\@startsection{section}{1}{0mm}%
                                {.2ex}%
                                {.2ex}%x
                                {\sffamily\small\bfseries}}
\renewcommand{\subsection}{\@startsection{subsection}{1}{0mm}%
                                {.2ex}%
                                {.2ex}%x
                                {\sffamily\bfseries}}
\renewcommand{\subsubsection}{\@startsection{subsubsection}{1}{0mm}%
                                {.2ex}%
                                {.2ex}%x
                                {\sffamily\footnotesize\bfseries}}

\begin{document}
\small
\begin{multicols*}{3}

{\sffamily\Large\bfseries Lamtez Cheat-sheet}

\section{Contract}
\begin{Verbatim}[frame=single]
<contract> ::=
  <type_declaration*>
  <storage_declaration*>
  <expression>
\end{Verbatim}
The expression must denote a function, taking a parameter and
returning a result. If types can't be innferred from the code, they must be annotated.


\section{Type declarations}

\subsection{Labelled sum types}
\begin{Verbatim}[frame=single]
type <name>=<Label_0>:<type_0>+...+<Label_n>:<type_n>
\end{Verbatim}

Union of named type cases, in arbitrary number, compiling into nested
\verb|Left/Right| cases. Two products or sum types cannot share a
common label. The value associated with a label can be omitted when it is
of type \verb|unit|.

The following sym types are built-in:
\begin{Verbatim}
type option t =  None | Some t
type bool     = False | True
type list a   =   Nil | Cons (a * list a)
\end{Verbatim}

\subsection{Labelled product types}
\begin{Verbatim}[frame=single]
type <name>=<Label_0>:<type_0>*...*<Label_n>:<type_n>
\end{Verbatim}

Record structures with named fields, in arbitrary number, compiling into
nested pairs. Two product or sum types cannot share a common label.

\subsection{Type aliases}
\begin{Verbatim}[frame=single]
type <name> <params>* = <type(params)>
\end{Verbatim}

\verb|type account = contract unit unit| is built-in.

\subsection{Unlabelled tuple types}
\begin{Verbatim}[frame=single]
(type_0 * ... * type_n)
\end{Verbatim}

Unlabelled tuples can be used as annotations without type
declarations, as a parameter for sum cases and product fields, or
named with a type alias.

\verb|type pair a b = (a*b)| is built-in.



\section{Storage declaration}
\begin{Verbatim}[frame=single]
@<name> :: <type> (= <expr>)?
\end{Verbatim}
Storage fields are persisted as the contract's storage. If every
storage field has an initialization value, the compiler can optionally
produce an initial store value litteral (option
\verb|--store-output|). Names can be capitalized or not.



\section{Expressions}

\subsection{Litterals}

{\bf Natural numbers:} \fbox{\tt~[0-9]+~}\\
e.g. \verb|1|, \verb|123456|

{\bf Signed integers:} \fbox{\tt~(+|-)[0-9]+~}\\
e.g. \verb|+1|, \verb|-2|, \verb|+0|.\\
Plus sign is mandatory for ints $\geq0$.\\
Beware that \verb|f+1| (without space) is parsed as \verb|f(+1)|, not \verb|f + 1|.

{\bf Tez:} \fbox{\tt~tz<d>*.<dd> | tz<d>+~}\\
where \verb|<d>| is a decimal digit, \verb|<dd>| a pair thereof.\\
e.g. \verb|tz1|, \verb|tz2.00|, \verb|tz.02|.

{\bf Key hashes:} \fbox{\tt~tz1<b58\_char>+~}\\
e.g. \verb|tz1MGxkasF5ABVVrxzxGbWo8wPg7EaLKn4RS|

{\bf Signatures:} \fbox{\tt~sig:[0-9a-f]+~}\\
e.g. \verb|sig:91b334be19d66d30205563b426c2f9b3|...

{\bf Date:} \fbox{
  \begin{minipage}{5.5cm}
  \tt dddd-dd-ddTdd:dd:ddZ |\\
    dddd-dd-ddTdd:dd:dd(+|-)dd(:dd?)
  \end{minipage}}\\
where \verb|d| are decimal digits.\\
e.g. \verb|1970-01-01T00:00:00Z|,\\
\verb|    2014-09-02T12:34:56+02:00|

{\bf String:} \fbox{\tt~"[\^{ }"]*"~}:\\
e.g. \verb|"foo"|, \verb|"hello \"world\""|.\\
Must fit on a single line, double quotes escaped with backslashes;
same escaped characters as Michelson.

\subsection{Composite types and collections}
\begin{Verbatim}[frame=single]
{ <Label_0>:<expr_0>,...,<Label_n>:<expr_n>} # Product
(<expr_0>, ..., <expr_n)                     # Tuple
(list <expr_0> ... <expr_n>)                 # List
(set <expr_0> ... <expr_n>)                  # Set
(map <expr_k0> <expr_v0>...<expr_kn> <expr_vn>) # Map
\end{Verbatim}

\subsection{Identifiers}
Identifiers start with a lowercase letter, can contain alphanumeric
characters, underscores and dashes.

\subsection{Functions}
\begin{Verbatim}[frame=single]
fun (<pattern>(::<type_param>)?)+ (::<type_result>)?:
   <expr>
\end{Verbatim}

As in ML, arguments are separated by spaces without mandatory
parentheses around them.  Function application binds tighter than
binary and unary operators (e.g. \verb|a b + c d| parses as
\verb|a(b) + c(d)|), but looser than product field accessors.

\begin{Verbatim}
f arg_0 arg_1 ... arg_n
f (g arg_g0 arg_g1) arg_f1
\end{Verbatim}

Functions are created with ``\verb|fun|'' standing for the $\lambda$
operator (backslash also accepted for Haskellers); parameters are
comma-separated between them, separated from the function body by a
colon; when parameter types can't be inferred, they must be annotated
with a double-colon \verb|::| sign. Optionally, the function result
can be annotated with a double colon after annotated parameters.

\begin{Verbatim}
fun p: p - 1
fun p0, p1: p0 * p1
fun p0 :: tez, p1 :: nat :: tez: p0 * p1
fun p :: unit: ()
fun p :: unit :: time: self-now
\end{Verbatim}

\subsection{Local variables}
\begin{Verbatim}[frame=single]
let <pattern> = <expr_0>; <expr_1>; ...; <expr_n>
\end{Verbatim}
Local variables are created with the \verb|let| keyword, either as
identifiers, or as binding patterns (nested product types).

\subsection{Binding patterns}
\begin{Verbatim}[frame=single]
  <id>
| _
| (<pattern_0>,...,<pattern_n>)
| {<Label_0>:<pattern_0>,...,<Label_n>:<pattern_n>}
\end{Verbatim}
When binding an identifier and a value (in function parameters, let
declarations and sum cases), patterns for product types and tuples can
be used, thus binding several variables simultaneously. Patterns can
be nested. They may not use sum types, as it would introduce the
possibility of runtime failures.

\begin{verbatim}
let (a, b) = (1, 2);
case p | Some {Lon: x, Lat: y}: ... end;
let norm = fun (x, y) :: int*int: x*x + y*y
\end{verbatim}

\subsection{Flow control}

\subsubsection{Sum case deconstruction}
\begin{Verbatim}[frame=single]
case <expr>
| <Label_0> <pattern_0>: <expr_0>
| ...
| <Label_n> <pattern_n>: <expr_n>
end
\end{Verbatim}

Labels can appear in any order, but all cases of one sum type must be
present. Patterns \verb|p_n| which are not used in \verb|e_n| can be
omitted. Booleans, options and lists are sum cases and can be
deconstructed with a sum type.

\subsubsection{If then else / switch}
\begin{Verbatim}[frame=single]
if
| <expr_cond_0>: <expr_if_true_0>
| ...
| <expr_cond_n>: <expr_if_true_n>
| else: <expr_if_everything_false>
end
\end{Verbatim}

The code \verb|<expr_if_true_i>| corresponding to the first true
\verb|<expr_cond_i>| is evaluated. \verb|else| clause may be
ommitted if the result type is \verb|unit|.

\subsection{Field access}
\begin{Verbatim}[frame=single]
<expr>.<d>                # Tuple field access
<expr>.<Label>            # Product type field access
<expr> <- <Label>: <expr> # Product type field update
@<name>                   # Persistent field access
@<name> <- <expr>;        # Persistent field update
<Label> <expr>            # Sum type constructor
\end{Verbatim}
Tuple fields are accessed with a dot followed by the field number, 0-indexed.

Labelled product fiels are accessed with a dot followed by the field label.

Persistent storage fields are accessed with ``\verb|@|'' followed by
the field name.  They can be updated
with \verb|@<name> <- <expr>;|,
but such operations cannot occur in a function, a litteral product or
a function argument.

Sum types are constructed with the case label followed by the
associated value; the value can be omitted when its type is
\verb|unit|.

\begin{Verbatim}
let t = ("a","b","c"); t.1;    # Tuple access
let p = {X:45,Y: 1}; p.X;      # Product access
let s :: option nat = Some 42; # Sum constr.
let x = not @field;            # store access
@field <- x                    # store update
\end{Verbatim}

\subsection{Type annotations}
\begin{Verbatim}[frame=single]
<expr> :: <type>
\end{Verbatim}

Michelson doesn't support polymorphic types. To prevent Lamtez from guessing
such types, as well as to improve contract readbility, you might have to add
type annotations.

\vfill\null\columnbreak

\subsection{Operations}
\subsubsection{Binary infix operators}
in decreasing order of precedence:
\begin{itemize}
\itemsep0em
\item \verb|a * b| and \verb|a / b| (euclidian division);
\item \verb|a + b| and \verb|a - b|;
\item bit shifting \verb|a << b| and \verb|a >> b|;
\item comparisons \verb|a < b|, \verb|a <= b|, \verb|a = b|, \verb|a != b|,
  \verb|a > b|, \verb|a >= b|;
\item logical and bitwise conjunction \verb|a && b|;
\item logical  disjunctions \verb%a || b% and \verb|a ^^ b|.
\end{itemize}

Beware of spaces around addition and substraction: ``\verb|-|'' is a
valid identifier character, and both characters can be interpreted as
prefix sign for a signed literal integer.


\subsubsection{Unary operators}
\verb|-<expr>|, \verb|not <expr>|.\\
Precedence higher than binary ops, lower than field accessors and
function applications.

\subsubsection{Operands and result types}

Operator types and semantics are lifted from Michelson:
\begin{itemize}
\itemsep0em
\item \verb|nat| and \verb|int| can be added/substracted/multiplied
  together, and result in a signed \verb|int|, except for \verb|nat+nat| and
  \verb|nat*nat| which result in \verb|nat|.
\item As in Michelson, divisions are euclidian, return an (integer
  division $*$ natural reminder) option pair (\verb|None| in case of
  division by 0).
\item \verb|nat| can be added/substracted with \verb|time|, \verb|tez|
  can be added/substracted together.
\item \verb|tez| can be multiplied by \verb|nat|, divided together or
  by a \verb|nat|.
\item Logical operators work on \verb|bool| as well as bitwise on \verb|nat|.
\item comparison operators work on every type (both operands must have
  the same type, though).
\end{itemize}

\vfill\null\columnbreak

\section{Primitives}

\subsection{Current contract context}
\begin{Verbatim}
val fail          :: fail # FAIL
val self-amount   :: tez  # AMOUNT
val self-balance  :: tez  # BALANCE
val self-now      :: time # NOW
val self-contract :: $\forall$p,r: contract p r # SELF
val self-source   :: $\forall$p,r: contract p r # SOURCE
val self-steps-to-quota :: nat # STEPS_TO_QUOTA
\end{Verbatim}

\subsection{Contract management}
\begin{lstlisting}
val contract-call :: $\forall$p,r:
  contract p r $\rightarrow$ p $\rightarrow$ tez $\rightarrow$
  r # TRANSFER_TOKENS
val contract-create :: $\forall$p,s,r:
  key $\rightarrow$ option key $\rightarrow$ bool $\rightarrow$ bool $\rightarrow$
  tez $\rightarrow$ (p*s$\rightarrow$r*s) $\rightarrow$ storage $\rightarrow$
  contract p r # CREATE_CONTRACT
val contract-create-account ::
  key $\rightarrow$ option key $\rightarrow$ bool $\rightarrow$ tez $\rightarrow$
  account # CREATE_ACCOUNT
val contract-get ::
  key $\rightarrow$ account # DEFAULT_ACCOUNT
val contract-manager :: $\forall$p, s:
  contract p s $\rightarrow$ key  # MANAGER
\end{lstlisting}

\verb|contract-call| cannot be used in a function / tuple / product / collection.\\

\subsection{Cryptography}
\begin{lstlisting}
val crypto-check :: key $\rightarrow$ sig $\rightarrow$ string $\rightarrow$ bool
val crypto-hash  :: $\forall$a: a $\rightarrow$ string
\end{lstlisting}


\subsection{Collections}

\begin{lstlisting}
val list-map   :: $\forall$a,b: list a $\rightarrow$ (a$\rightarrow$b) $\rightarrow$ list b
val list-reduce:: $\forall$a,acc:
    list a $\rightarrow$ acc $\rightarrow$ (a$\rightarrow$acc$\rightarrow$acc) $\rightarrow$ acc
val list-size  :: $\forall$a: list a $\rightarrow$ nat
val set-map    :: $\forall$a,b: set a $\rightarrow$ (a$\rightarrow$b) $\rightarrow$ set b
val set-mem    :: $\forall$elt: set elt $\rightarrow$ elt $\rightarrow$ bool
val set-reduce :: $\forall$elt acc:
    set elt $\rightarrow$ acc $\rightarrow$ (elt$\rightarrow$acc$\rightarrow$acc) $\rightarrow$ acc
val set-update :: $\forall$elt: elt $\rightarrow$ bool $\rightarrow$ set elt
val set-size   :: $\forall$elt: set elt $\rightarrow$ nat
val map-get    :: $\forall$k,v: k $\rightarrow$ map k v $\rightarrow$ option v
val map-map    :: $\forall$k,v0,v1:
    map k v0 $\rightarrow$ (k$\rightarrow$v0$\rightarrow$v1) $\rightarrow$ map k v1
val map-mem    :: $\forall$k,v: k $\rightarrow$ map k v $\rightarrow$ bool
val map-reduce :: $\forall$k,v,acc:
    map k v $\rightarrow$ acc (k$\rightarrow$v$\rightarrow$acc$\rightarrow$acc) $\rightarrow$ $\rightarrow$ acc
val map-update :: $\forall$k,v:
    k $\rightarrow$ option v $\rightarrow$ map k v $\rightarrow$ map k v
val map-size   :: $\forall$k,v: map k v $\rightarrow$ nat
\end{lstlisting}

\end{multicols*}
\end{document}
